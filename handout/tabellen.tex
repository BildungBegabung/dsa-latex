\section{Tabellen}
Tabellen befinden sich in Tabellenumgebungenen, die
aus mehreren Elementen bestehen. Sie haben:

\begin{itemize}
 \item eine Überschrift \lstinline$\caption{}$
 \item ein Label (auf alle(!) Tabellen muss im Text verwiesen werden)
       \lstinline$\label{}$
 \item die Tabelle selbst
 \begin{itemize}
  \item Anzahl der Spalten, die jeweils linksbündig \lstinline$l$, zentriert
        \lstinline$c$ oder rechtsbündig \lstinline$r$ sein können
  \item Spalten werden mit \lstinline$&$ getrennt
  \item am Ende jeder Zeile steht \lstinline$\\$
 \end{itemize}
\end{itemize}

\begin{lstlisting}
\begin{table}[h]
 \caption{Dieses ist unsere Beispieltabelle. Der Inhalt ist weder relevant noch
          zutreffend.}
 \centering
 \begin{tabular}{lcr} % drei Spalten
  \toprule
   Kurs      & blaue Augen & braune Augen\\
  \midrule
   5.1       & 4           & 10\\
   5.2       & 7           & 9\\
   5.3       & 15          & 1\\
  \bottomrule
 \end{tabular}
 \label{table:beispiel}
\end{table}
\end{lstlisting}

\begin{table}[h]
 \caption{Dieses ist unsere Beispieltabelle. Der Inhalt ist weder relevant noch
          zutreffend.}
 \centering
 \begin{tabular}{lcr} % drei Spalten
  \toprule
   Kurs      & blaue Augen & braune Augen\\
  \midrule
   5.1       & 4           & 10\\
   5.2       & 7           & 9\\
   5.3       & 15          & 1\\
  \bottomrule
 \end{tabular}
 \label{table:beispiel}
\end{table}

\danger Für den Fall der SchülerAkademie muss \lstinline$\table$ durch
        \textbf{\textbackslash dsatable} ersetzt werden. Desweiteren ist die
        Option \lstinline$[h]$ im DSA-template nicht zulässig. Sie sollte
        einfach weggelassen werden.
