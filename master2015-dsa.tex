\documentclass{newlayout}
%Bitte hier den enstprechenden Ort einsetzen z.B. Braunschweig und die Akademienummer
\Akademie{Ort}{2014}{1}

\usepackage[english,ngerman]{babel}
\usepackage{misc}
\usepackage{multicol}

%\usepackage{amsmath}%wird automatisch durch newlayout.cls geladen
\usepackage{amsfonts}

\usepackage{url}
\def\UrlBreaks{\do\a\do\b\do\c\do\d\do\e\do\f\do\g\do\h\do\i\do\j\do\k\do\l%
\do\m\do\n\do\o\do\p\do\q\do\r\do\s\do\t\do\u\do\v\do\w\do\x\do\y\do\z\do\0%
\do\1\do\2\do\3\do\4\do\5\do\6\do\7\do\8\do\9\do\-\do\_\do\/\do\%}
\urlstyle{same}

% hinzugef�gt, um Fehler 'pdfTeX error (font expansion): auto expansion is only possible with scalable' zu vermeiden
%\usepackage{lmodern}
\setkomafont{descriptionlabel}{\normalfont\bfseries}
\addtokomafont{paragraph}{\normalfont}
\usepackage{footnote}
\usepackage[flushmargin,hang,ragged]{footmisc}
\deffootnote{1em}{1em}{%
\textsuperscript{\thefootnotemark\ }
}
%\setlength{\abovedisplayskip}{5pt}
%\setlength{\belowdisplayskip}{5pt}


%%%%%Mathe-Definitionen
\newtheorem{Def}{Definition}
\newtheorem{Sat}{Satz}
\newtheorem{Bew}{Beweis}

\setlength\abovedisplayshortskip{0pt}
\setlength\belowdisplayshortskip{0pt}
\setlength\abovedisplayskip{3pt}
\setlength\belowdisplayskip{3pt}
%%%%Ende Mathe-Definitionen

\begin{document}

 %   \input{titel}
 \setcounter{page}{3}

\setcounter{tocdepth}{1}
 \tableofcontents

   \setcounter{secnumdepth}{1}


\setcounter{page}{7}
\setcounter{chapter}{0}

%Angabe, bis zu welcher Stufe die sections im Text nummeriert werden sollen.
      \settocdepth{2}



\course{1}{Kurstitel eingeben}%%% 
\begin{coursetitle}
  \centerline{Kurstitel eingeben} 
  \bigskip
  \Large \centerline{Kursuntertitel eingeben}
  \bigskip
 \includegraphics[width=.9\columnwidth]{Titelbild-fehlt.png}
 \label{fig:meinbild}
  \bigskip
\end{coursetitle}
%\begin{dsafigure}
%\begin{center}
%\includegraphics[width=.9\columnwidth]{Titelbild-fehlt.png}
%\caption{meine Bildunterschrift}
%label{fig:meinbild}
%\end{center}
%\end{dsafigure}




\section{Aus der Kursbeschreibung}
\authors{Vorname Nachname KL, Vorname, Nachname KL}

Jeder Kurs beginnt mit einem Teil "`1 Aus der Kursbeschreibung"', entnommen aus dem Programmheft.
Hier sollte nur der den Inhalt des Kurses beschreibende
Teil der Kursbeschreibung erscheinen, nicht
die Erwartungen an die Teilnehmenden.




Gegen Ende des Kurses können die Teilnehmenden je nach Interesse weitere Themen diskutieren. Es kann zum Beispiel das ${H_2}^+$-Ion als ein
%\input{Kapitel-2-Kapitel�berschrift}
%\input{Kapitel-3-Kapitel�berschrift}
%usw.
\end{document}
